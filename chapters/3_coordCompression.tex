\section{Koordinatenkompression}
\begin{frame}
	\begin{center}
		\textbf{\LARGE{Koordinatenkompression}}
	\end{center}
\end{frame}

\subsection{Was ist das?}
\begin{frame}
	\frametitle{{Koordinatenkompression}}
	\begin{itemize}
		\item Abbilden von gegebenen Koordinaten auf geschicktere Koordinaten
		\pause
		\item Was hei"st geschickt?
		\pause
		\begin{itemize}
			\item Der Algorithmus soll dadurch schneller werden
			\pause
			\item Das Ergebnis darf nicht verfälscht werden
		\end{itemize}
		\pause
		\item Also: Irrelevante Bereiche aussparen
	\end{itemize}
\end{frame}

\subsection{Bei Fl\"achenberechnungen}
\begin{frame}
	\frametitle{{Fl\"achenberechnung}}
	\begin{block}{Das Problem}
	\begin{itemize}
		\pause
		\item{Eingabe:}
		\pause
		\begin{itemize}
			\item Eine 2-dimensionale Figur
			\item Bestehend aus Rechtecken
			\begin{itemize}
				\item Punkt unten links
				\item Punkt oben rechts
				\item Seiten parallel zu den Koordinatenachsen
			\end{itemize}
		\end{itemize}
		\pause
		\item{Ausgabe:}
		\pause
		\begin{itemize}
			\item{Fl\"acheninhalt der Figur}
		\end{itemize}
	\end{itemize}
	\end{block}
\end{frame}
\begin{frame}
	\frametitle{{Fl\"achenberechnung, erster Ansatz}}
	\begin{itemize}
		\item Iteration \"uber alle Datenfelder
		\pause
		\item Aufsummieren aller belegter Datenfelder
	\end{itemize}
\end{frame}
\begin{frame}
	\frametitle{{Fl\"achenberechnung, erster Ansatz}}
	\begin{itemize}
		\item Iteration \"uber alle Datenfelder
		\item Aufsummieren aller belegter Datenfelder
	\end{itemize}
	\begin{figure}
		\begin{center}
			\includegraphics[scale=0.50]{bilder/kidding.png}		
		\end{center}
	\end{figure}
\end{frame}
\begin{frame}
	\frametitle{{Fl\"achenberechnung mit Koordinatenkompression}}
		\pause
	\begin{itemize}
		\item Vorarbeit
		\pause
		\begin{itemize}
			\item Kompression auf ein Feld maximal der Gr\"o"se $\mathcal (2 \cdot \left|Rechtecke\right|)\times(2 \cdot \left|Rechtecke\right|)$
			\pause
			\item Kleiner, falls sich Rechtecke eine x- oder y-Stelle teilen
			\pause
			\item Merken der tats\"achlichen Gr\"o"se
		\end{itemize}
		\pause
		\item Nun wie gehabt belegte Felder z\"ahlen
			\pause
		\begin{itemize}
			\item dabei deren tats\"achliche Fl\"achen aufsummieren
		\end{itemize}
	\end{itemize}
\end{frame}

\subsection{Weitere Anwendungsgebiete}
\begin{frame}
	\frametitle{{Weitere Anwendungsgebiete von Koordinatenkompression}}
	\begin{itemize}
		\item Volumenberechnung
		\pause
		\item Bereichssuchen
	\end{itemize}
\end{frame}