\section{Koordinatenkompression}
\begin{frame}
	\begin{center}
		\textbf{Koordinatenkompression}
	\end{center}
\end{frame}

\subsection{Was ist das?}
\begin{frame}
	\frametitle{{Koordinatenkompression}}
	\begin{itemize}
		\item Abbilden von gegebenen Koordinaten auf geschicktere Koordinaten
		\pause
		\item Was hei"st geschickt?
		\pause
		\begin{itemize}
			\item TODO: Weiter ausf\"uhren!
		\end{itemize}
		\pause
		\item Dadurch k\"onnen bestimmte Algorithmen beschleunigt werden
	\end{itemize}
\end{frame}

\subsection{Anwendungsgebiete}
\begin{frame}
	\frametitle{{Anwendungsgebiete von Koordinatenkompression}}
	\begin{itemize}
		\item Volumen und Fl\"achenberechnung
		\pause
		\item Bereichssuchen
		\pause
		\item TODO: Mehr finden, oder die beiden Punkte mehr ausschmücken!
		\begin{itemize}
			\item (TODO) Wodurch genau profitieren diese Algorithmen?
			\item (TODO) Beispiel?
		\end{itemize}
	\end{itemize}
\end{frame}