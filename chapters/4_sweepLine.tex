\section{Sweep}
\begin{frame}
	\begin{center}
		\textbf{Sweep}
	\end{center}
\end{frame}

\subsection{Line Segment Intersection}
\begin{frame}
	\frametitle{{Line Segment Intersection}}
	\begin{block}{Das Problem}
	\begin{itemize}
		\pause
		\item{Gegeben:}
		\pause
		\begin{itemize}
			\item{2D Koordinatensystem}
			\pause
			\item{n Liniensegmente}
		\end{itemize}
		\pause
		\item{Gesucht:}
		\pause
		\begin{itemize}
			\item{Anzahl der Schnittpunkte}
		\end{itemize}
	\end{itemize}
	\end{block}
\end{frame}
\begin{frame}
	\frametitle{{Line Segment Intersection}}
	\begin{itemize}
		\item \textbf{Naiv}
		\begin{itemize}
			\pause
			\item{Vergleiche jedes Segment mit jedem anderen}
			\pause
			\item{Resultierende Laufzeit: $\mathcal O(n²)$}
		\end{itemize}
	\end{itemize}
\end{frame}
\begin{frame}
	\frametitle{{Line Segment Intersection}}
	\begin{itemize}
		\item \textbf{Weitere \"Uberlegungen}
		\pause
		\item{Betrachten der Situation in der Stelle a}
		\pause
		\item{Nur direkt benachbarte Segmente k\"onnen sich schneiden}
		\pause
		\item{Dieser Status \"andert sich nur dann, wenn}
		\begin{itemize}
			\pause
			\item{Ein neues Segment beginnt}
			\pause
			\item{Ein Segment endet}
			\pause
			\item{Wenn sich zwei Segmente schneiden}
		\end{itemize}
	\end{itemize}
\end{frame}
\begin{frame}
	\frametitle{{Line Segment Intersection}}
	\begin{itemize}
		\item \textbf{Sweep Line Algorithmus}
	\end{itemize}
\end{frame}

\subsection{Allgemeine Eigenschaften}
\begin{frame}
	\frametitle{{Allgemeine Eigenschaften von Sweep Algorithmen}}
	\begin{itemize}
		\item Anwendbar auf Probleme beliebig hoher Dimension
		\begin{itemize}
			\pause
			\item{Im 2-Dimensionalen Raum}
			\begin{itemize}
				\item{Gerade, die über sich über die Ebene bewegt}
			\end{itemize}
			\pause
			\item{Im 3-Dimensionalen Raum}
			\begin{itemize}
				\item{Ebene, die über sich durch den K\"orper bewegt}
			\end{itemize}
			\pause
		\end{itemize}
	\end{itemize}
\end{frame}
\begin{frame}
	\frametitle{{Allgemeine Eigenschaften von Sweep Algorithmen}}
	\begin{itemize}
		\item{Sweep Algorithmen verwenden}
		\pause		
		\begin{itemize}
			\item{Einen Status}
			\pause
			\item{Im Beispiel:}
			\begin{itemize}
				\item{Noch nicht abgeschlossene Segmente}
				\pause
				\item{Und deren Sortierung}
			\end{itemize}
			\pause
			\item{Eine Event Queue}
			\pause
			\item{Im Beispiel:}
			\begin{itemize}
				\item{Endpunkte von Segmenten}
				\pause
				\item{Schnittpunkte von Segmenten}
			\end{itemize}
		\end{itemize}
	\end{itemize}
\end{frame}

\subsection{Weitere Algorithmen}
\begin{frame}
	\frametitle{{Weitere Sweep Algorithmen}}
	\begin{itemize}
		\pause
		\item Fortune's algorithm
		\begin{itemize}
			\item Sweep Line Algorithmus zur Erstellung eines Voronoi-Diagramms
			\pause			
			\item M\"ogliches Anwendungsgebiet: Robotik, Hindernissen fern bleiben
		\end{itemize}
		\pause
		\item Boolsche Operationen auf Polygone
		\begin{itemize}
			\item M\"ogliche Anwendungsgebiete: Computergraphik oder CAD-Systeme
		\end{itemize}
		\pause
		\item Closest Pair
		\pause
		\item Viele mehr!
	\end{itemize}
\end{frame}
